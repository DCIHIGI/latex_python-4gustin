\documentclass[10pt,a4paper]{article}
\usepackage[utf8]{inputenc}
\usepackage[spanish]{babel}
\usepackage{amsmath}
\usepackage{amsfonts}
\usepackage{amssymb}
\usepackage[left=2cm,right=2cm,top=2cm,bottom=2cm]{geometry}
\author{Agustin Nelson Medina Colmenero}
\title{Primer Documento \LaTeX}

\begin{document}

\maketitle

\begin{abstract}
El documento presenta algunas de las ecuaciones más importantes de las ramas de física y matemáticas dado el impacto que han tenido el el mundo según \cite{Ec}.
Además se incluye un listado de las físicos y matemáticos que han hecho grandes aportaciones a sus respectivas áreas, lo mencionado anteriormente se realizó con la finalidad de practicar el tipeo de ecuaciones y la construcción de listas en el procesador de texto \LaTeX . 
\end{abstract}


%%%%%%%%%%%%%%%%%%%%%%%%%%%%%%%%%%%%%%%%%%%%%%%%%%%%%%%%%%%%%%%%%%%%%%%%%%%%%%%%%%%%%%%%%%%%%%%%%%%%%%%%%

\section*{Ecuaciones más importantes de la física}

\subsubsection*{Ley de la Gravedad de Newton}
\begin{equation}
F = G \frac{m_1 m_2}{r^2}
\end{equation}


\subsubsection*{Ecuaciones de Maxwell}
\begin{equation}
\nabla \cdot \textbf{E} = \frac{\rho}{\varepsilon_0} 
\end{equation}

\begin{equation}
\nabla \cdot \textbf{B} = 0 
\end{equation}

\begin{equation}
\nabla \times \textbf{E} = - \frac{\partial \textbf{B}}{\partial t}
\end{equation}

\begin{equation}
\nabla \times \textbf{B} =
 \mu_0 \textbf{j} +\mu_0 \varepsilon _0 \frac{\partial \textbf{E}}{\partial t}
\end{equation}


\subsubsection*{Ecuación de Navier-Stokes}
\begin{equation}
\rho (\frac{\partial v}{\partial t} + v \cdot \nabla v ) = - \nabla p + \nabla \cdot T + f
\end{equation}


\subsubsection*{Ecuación de Onda}
\begin{equation}
\frac{\partial ^2 u}{\partial t^2} = c^2 \frac{\partial ^2 u}{\partial x^2}
\end{equation}


%%%%%%%%%%%%%%%%%%%%%%%%%%%%%%%%%%%%%%%%%%%%%%%%%%%%%%%%%%%%%%%%%%%%%%%%%%%%%%%%%%%%%%%%%%%%%%%%%%%%%%%

\section*{Ecuaciones más importantes de las matemáticas}


\subsubsection*{Teorema de Pitágoras}
\begin{equation}
a^2 + b^2 = c^2
\end{equation}


\subsubsection*{Logaritmos}
\begin{equation}
log_b(x y) = log_b(x) +log_b(y) 
\end{equation}


\subsubsection*{Distribución normal}
\begin{equation}
\Phi (x) = \frac{1}{\sqrt{2 \pi \sigma}} e^{\frac{(x- \mu)^2}{2 \sigma ^2}}
\end{equation}


\subsubsection*{Transformada de Fourier}
\begin{equation}
g(\xi)= \frac{1}{\sqrt{2 \pi}} \int_{- \infty}^{+ \infty} f(x) e^{- i \xi x} dx
\end{equation}

%%%%%%%%%%%%%%%%%%%%%%%%%%%%%%%%%%%%%%%%%%%%%%%%%%%%%%%%%%%%%%%%%%%%%%%%%%%%%%%%%%%%%%%%%%%%%%%%%%

\section*{Físicos que realizaron contribuciones importantes según \cite{physics}}

\begin{enumerate}
	\item Isaac Newton
	\item Albert Einstein
	\item Galieo Gailei
	\item Sthepen Hawking
	\item Murray Gell-Mann
	\item John Cockcroft
	\item Joseph J. Thomson
	\item Chandrasekhara Raman
	\item Arthur Compton
	\item Marie Curie
	\item Michael Faraday
	\item Niels Bohr
	\item Paul Dirac
	\item James Clerk Maxwell
	\item Max Planck
\end{enumerate}

%%%%%%%%%%%%%%%%%%%%%%%%%%%%%%%%%%%%%%%%%%%%%%%%%%%%%%%%%%%%%%%%%%%%%%%%%%%%%%%%%%%%%%%%%%%%%%%%

\section*{Matemáticos relevantes \cite{math}}

\begin{itemize}
	\item Pitágoras
	\item Jean-Baptiste Joseph Fourier
	\item Isaac Newton
	\item Alan Turing
	\item Kurt Gödel
	\item Emmy Noether
	\item René Descartes
	\item Al Juarismi
\end{itemize}

%%%%%%%%%%%%%%%%%%%%%%%%%%%%%%%%%%%%%%%%%%%%%%%%%%%%%%%%%%%%%%%%%%%%%%%%%%%%%%%%%%%%%%%%%%%%%%%%%%%%%%%%%%

\begin{thebibliography}{2}
 
	\bibitem{Ec} Stewart, I., Fernández, L. S. (2013). 
\textit{17 ecuaciones que cambiaron el mundo.} Planeta.

	\bibitem{physics} Cajal, A. (2019, 11 marzo).\textit{ Los 30 Físicos Más Famosos e Importantes de la Historia. Lifeder.} \textbf{https://www.lifeder.com/fisicos-famosos/}

\bibitem{math} García, P. (2019, 5 marzo). \textit{Grandes momentos de la historia de las matemáticas.} OpenMind. \textbf{https://www.bbvaopenmind.com/ciencia/matematicas/grandes-momentos-la-historia-las-matematicas/
}

 
\end{thebibliography}

\end{document}